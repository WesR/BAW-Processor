\documentclass[a4paper,14pt]{article}
\usepackage[utf8]{inputenc}
\usepackage[margin=1in]{geometry}
\usepackage{caption}
\usepackage{multirow}
\usepackage{lscape}
\usepackage{hyperref}

%opening
\title{The BAW Instruction Set Manual}
\author{Bryant Herren, Wesley Ring, Austin Waddell}
\date{}

\begin{document}
\null  % Empty line
\nointerlineskip  % No skip for prev line
\vfill
\let\snewpage \newpage
\let\newpage \relax
\maketitle
\let \newpage \snewpage
\vfill 
\newpage

{\hypersetup{linktoc=all,hidelinks}
\tableofcontents
}
\newpage

\section{Introduction}
BAW is the instruction set for the processor we are building for ECGR 3183. It is a floating point co-processor with a single-cycle architecture, and a pipelined architecture. For the pipelined architecture, we implement two branch prediction algorithms (1 static and 1 dynamic) as well as no branch prediction.
\newline\newline
We will be providing:
\begin{itemize}
    \setlength{\parskip}{0pt}
    \setlength{\itemsep}{0pt plus 1pt}
    \item the documented ISA
    \item architecture and controller design (units, diagrams, etc)
    \item VHDL/C++/other simulation
    \item performance results and discussion for pipelined vs. unpipelined approaches
\end{itemize}


\subsection{System Parameters}
\begin{itemize}
    \setlength{\parskip}{0pt}
    \setlength{\itemsep}{0pt plus 1pt}
    \item 32 Bits
    \item Data is stored using IEEE single-precision floating point numbers.
    \item There are 16 Registers
    \item Timings:
    \begin{itemize}
        \setlength{\parskip}{0pt}
        \item Clock cycle (pipelined): 100ns
        \item Register Read/Write: 100ns
        \item Memory Read/Write: 300ns
        \item Single ALU Op: 200ns
    \end{itemize}
\end{itemize}

\subsection{Memory}
The system includes a data memory addressed 0-1023 and 16 Floating Point registers (Referenced as X0-X15). Each memory location and register uses a 32-bit value. The simulation can read an input file containing the operational parameters, code, and memory contents (there is an assembler).
\subsection{Additional Features}
\begin{itemize}
    \setlength{\parskip}{0pt}
    \setlength{\itemsep}{0pt plus 1pt}
    \item FP Multiply by -1, 1, or 0 takes 1 cycle
    \item FP Multiply by power of 2 takes 2 cycles
    \item Condition Codes: ZNV (zero, negative, overflow)
    \begin{itemize}
        \setlength{\parskip}{0pt}
        \setlength{\itemsep}{0pt plus 1pt}
        \item All condition codes are set as needed on arithmetic operations (pay special attention to the FP ALU operations and results)
    \end{itemize}
    \item The round, ceiling, and floor functions would round up to the nearest integer (not always a power of 2), expressing the result in floating point format.
\end{itemize}

\section{ISA}
\subsection{Introduction}
Our ISA is based off of the ARMv8 / LEGv8. Because of this, you will likely see similarity.

\subsection{Instruction Format}
We have three instruction formats, Register (R), Data (D), and Immediate (I). 
All processor instructions are 32 bits wide. Table \ref{table:format_descriptions} Contains information about the specific instruction formats.
\newline
Notes:
\begin{enumerate}
    \item The opcode is 8 bits long, allowing to be easily read in hex format.
    \item The remaining three bits of each instruction format are not used.
\end{enumerate}

\begin{table}
\centering
\caption{About the formats}
\label{table:format_descriptions}
\begin{tabular}{|l|l|l|}
\hline
Format & Description & Example \\ \hline
R (Register) & An instruction whose inputs and outputs are both registers & ADD X9, X21, X9 \\ \hline
D (Data) & Used when fetching or placing data in memory & LOAD X9, {[}X22, \#64{]} \\ \hline
I (Immediate) & An instruction with data in the instruction & POW X9, \#15 \\ \hline
\end{tabular}
\end{table}

Please see Table \ref{table:instruction_formats} for specific information on the Instruction Formats.

\begin{landscape}
\begin{table}
\setlength\tabcolsep{4pt}
\caption{Instruction Formats}
\label{table:instruction_formats}
\begin{tabular}{|l|l|l|l|l|l|l|l|l|llllllllllll|l|l|l|l|l|l|l|l|l|l|l|l|l|}
\hline
\multicolumn{1}{|r|}{Instruction Format} & 0 & 1 & 2 & 3 & 4 & 5 & 6 & 7 & \multicolumn{1}{l|}{8} & \multicolumn{1}{l|}{9} & \multicolumn{1}{l|}{10} & \multicolumn{1}{l|}{11} & \multicolumn{1}{l|}{12} & \multicolumn{1}{l|}{13} & \multicolumn{1}{l|}{14} & \multicolumn{1}{l|}{15} & \multicolumn{1}{l|}{16} & \multicolumn{1}{l|}{17} & \multicolumn{1}{l|}{18} & 19 & 20 & 21 & 22 & 23 & 24 & 25 & 26 & 27 & 28 & 29 & 30 & 31 & 32 \\ \hline
R (Register) & \multicolumn{8}{l|}{opcode (8 bits)} & \multicolumn{5}{l|}{Rm (5 bit)} & \multicolumn{7}{l|}{Empty (7 bit)} & \multicolumn{5}{l|}{Rn (5 bit)} & \multicolumn{5}{l|}{Rd (5 bit)} & \multicolumn{3}{l|}{\multirow{3}{*}{\begin{tabular}[c]{@{}l@{}}Empty \\ \\ (3 bit)\end{tabular}}} \\ \cline{1-31}
D (Data) & \multicolumn{8}{l|}{opcode (8 bits)} & \multicolumn{10}{l|}{Address (9 bit)} & \multicolumn{2}{l|}{op2 (2 bit)} & \multicolumn{5}{l|}{Rn (5 bit)} & \multicolumn{5}{l|}{Rd (5 bit)} & \multicolumn{3}{l|}{} \\ \cline{1-31}
I (Immediate) & \multicolumn{8}{l|}{opcode (8 bits)} & \multicolumn{12}{l}{Immediate Data (11 bit)} & \multicolumn{5}{l|}{Rn (5 bit)} & \multicolumn{5}{l|}{Rd (5 bit)} & \multicolumn{3}{l|}{} \\ \hline
\end{tabular}
\end{table}
\end{landscape}

\subsection{Instructions}

\subsubsection{Set}
\begin{table}[h]
\centering
\caption*{SET}
\begin{tabular}{llllll}
ASM & opcode & Format & Description & Operation & ALU Cycles \\ \hline
\multicolumn{1}{|c|}{SET} & \multicolumn{1}{c|}{0000 0001} & \multicolumn{1}{c|}{D} & \multicolumn{1}{c|}{Sets Ri to given floating point value} & \multicolumn{1}{c|}{Ri $\leftarrow$ FPvalue} & \multicolumn{1}{c|}{1} \\ \hline
\end{tabular}
\end{table}

\begin{itemize}
    \setlength{\parskip}{0pt}
    \setlength{\itemsep}{0pt plus 1pt}
    \setlength{\itemindent}{-4mm}
    \item[] \textbf{ASM Example:} Set Ri, \#FPvalue
\end{itemize}
\begin{itemize}
    \setlength{\parskip}{0pt}
    \setlength{\itemsep}{0pt plus 1pt}
    \setlength{\itemindent}{7mm}
    \item [\textbf{Flags}]
    \item Zero
    \item Negitive
    \item Error
\end{itemize}

\subsubsection{Load}
\begin{table}[h]
\centering
\caption*{LOAD}
\begin{tabular}{llllll}
ASM & opcode & Format & Description & Operation & ALU Cycles \\ \hline
\multicolumn{1}{|c|}{LOAD} & \multicolumn{1}{c|}{0000 0010} & \multicolumn{1}{c|}{D} & \multicolumn{1}{c|}{Copies Rj from memory and into Ri} & \multicolumn{1}{c|}{Ri $\leftarrow$ M[Rj]} & \multicolumn{1}{c|}{1} \\ \hline
\end{tabular}
\end{table}

\begin{itemize}
    \setlength{\parskip}{0pt}
    \setlength{\itemsep}{0pt plus 1pt}
    \setlength{\itemindent}{-4mm}
    \item[] \textbf{ASM Example:} Load Ri, Rj
\end{itemize}
\begin{itemize}
    \setlength{\parskip}{0pt}
    \setlength{\itemsep}{0pt plus 1pt}
    \setlength{\itemindent}{7mm}
    \item [\textbf{Flags}]
    \item Zero
    \item Negitive
    \item Error
\end{itemize}





\section{Architecture}
\subsection{Datapath}
\subsection{controller}
\section{VHDL Description}
\section{Testing}
\section{Conclusion}
\end{document}
