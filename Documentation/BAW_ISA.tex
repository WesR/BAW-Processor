\documentclass[a4paper,14pt]{article}
\usepackage[utf8]{inputenc}
\usepackage[margin=1in]{geometry}
\usepackage{caption}
\usepackage{multirow,array,varwidth}
\usepackage{lscape}
\usepackage{hyperref}

%opening
\title{The BAW Instruction Set Manual}
\author{Bryant Herren, Wesley Ring, Austin Waddell}
\date{}

\begin{document}

\newcommand\DescEntry[1]{%
  \multirow{1}*{%
    \begin{varwidth}{13em}% --- or minipage, if you prefer a fixed width
    \flushright #1%
    \end{varwidth}}}

\newcommand\Tstrut{\rule{0pt}{2.6ex}}       % "top" strut
\newcommand\Bstrut{\rule[-0.9ex]{0pt}{0pt}} % "bottom" strut
\newcommand{\TBstrut}{\Tstrut\Bstrut} % top&bottom struts
    
\null  % Empty line
\nointerlineskip  % No skip for prev line
\vfill
\let\snewpage \newpage
\let\newpage \relax
\maketitle
\let \newpage \snewpage
\vfill 
\newpage

{\hypersetup{linktoc=all,hidelinks}
\tableofcontents
}
\newpage

\section{Introduction}
BAW is the instruction set for the processor we are building for ECGR 3183. It is a floating point co-processor with a single-cycle architecture, and a pipelined architecture. For the pipelined architecture, we implement two branch prediction algorithms (1 static and 1 dynamic) as well as no branch prediction.
\newline\newline
We will be providing:
\begin{itemize}
    \setlength{\parskip}{0pt}
    \setlength{\itemsep}{0pt plus 1pt}
    \item the documented ISA
    \item architecture and controller design (units, diagrams, etc)
    \item VHDL/C++/other simulation
    \item performance results and discussion for pipelined vs. unpipelined approaches
\end{itemize}


\subsection{System Parameters}
\begin{itemize}
    \setlength{\parskip}{0pt}
    \setlength{\itemsep}{0pt plus 1pt}
    \item 32 Bits
    \item Data is stored using IEEE single-precision floating point numbers.
    \item There are 16 Registers
    \item Timings:
    \begin{itemize}
        \setlength{\parskip}{0pt}
        \item Clock cycle (pipelined): 100ns
        \item Register Read/Write: 100ns
        \item Memory Read/Write: 300ns
        \item Single ALU Op: 200ns
    \end{itemize}
\end{itemize}

\subsection{Memory}
The system includes a data memory addressed 0-1023 and 16 Floating Point registers (Referenced as X0-X15). Each memory location and register uses a 32-bit value. The simulation can read an input file containing the operational parameters, code, and memory contents (there is an assembler).
\subsection{Additional Features}
\begin{itemize}
    \setlength{\parskip}{0pt}
    \setlength{\itemsep}{0pt plus 1pt}
    \item FP Multiply by -1, 1, or 0 takes 1 cycle
    \item FP Multiply by power of 2 takes 2 cycles
    \item Condition Codes: ZNV (zero, negative, overflow)
    \begin{itemize}
        \setlength{\parskip}{0pt}
        \setlength{\itemsep}{0pt plus 1pt}
        \item All condition codes are set as needed on arithmetic operations (pay special attention to the FP ALU operations and results)
    \end{itemize}
    \item The round, ceiling, and floor functions would round up to the nearest integer (not always a power of 2), expressing the result in floating point format.
\end{itemize}

\section{ISA}
\subsection{Introduction}
Our ISA is based off of the ARMv8 / LEGv8. Because of this, you will likely see similarity.

\subsection{Instruction Format}
We have three instruction formats, Register (R), Data (D), and Immediate (I). 
All processor instructions are 32 bits wide. Table \ref{table:format_descriptions} Contains information about the specific instruction formats.
\newline
Notes:
\begin{enumerate}
    \item The opcode is 8 bits long, allowing to be easily read in hex format.
    \item The remaining three bits of each instruction format are not used.
\end{enumerate}

\begin{table}
\centering
\caption{About the formats}
\label{table:format_descriptions}
\begin{tabular}{|l|l|l|}
\hline
Format & Description & Example \\ \hline
R (Register) & An instruction whose inputs and outputs are both registers & ADD X9, X21, X9 \\ \hline
D (Data) & Used when fetching or placing data in memory & LOAD X9, {[}X22, \#64{]} \\ \hline
I (Immediate) & An instruction with data in the instruction & POW X9, \#15 \\ \hline
\end{tabular}
\end{table}

Please see Table \ref{table:instruction_formats} for specific information on the Instruction Formats.

\begin{landscape}
\begin{table}
\setlength\tabcolsep{4pt}
\caption{Instruction Formats}
\label{table:instruction_formats}
\begin{tabular}{|l|l|l|l|l|l|l|l|l|llllllllllll|l|l|l|l|l|l|l|l|l|l|l|l|l|}
\hline
\multicolumn{1}{|r|}{Instruction Format} & 0 & 1 & 2 & 3 & 4 & 5 & 6 & 7 & \multicolumn{1}{l|}{8} & \multicolumn{1}{l|}{9} & \multicolumn{1}{l|}{10} & \multicolumn{1}{l|}{11} & \multicolumn{1}{l|}{12} & \multicolumn{1}{l|}{13} & \multicolumn{1}{l|}{14} & \multicolumn{1}{l|}{15} & \multicolumn{1}{l|}{16} & \multicolumn{1}{l|}{17} & \multicolumn{1}{l|}{18} & 19 & 20 & 21 & 22 & 23 & 24 & 25 & 26 & 27 & 28 & 29 & 30 & 31 & 32 \\ \hline
R (Register) & \multicolumn{8}{l|}{opcode (8 bits)} & \multicolumn{5}{l|}{Rm (5 bit)} & \multicolumn{7}{l|}{Empty (7 bit)} & \multicolumn{5}{l|}{Rn (5 bit)} & \multicolumn{5}{l|}{Rd (5 bit)} & \multicolumn{3}{l|}{\multirow{3}{*}{\begin{tabular}[c]{@{}l@{}}Empty \\ \\ (3 bit)\end{tabular}}} \\ \cline{1-31}
D (Data) & \multicolumn{8}{l|}{opcode (8 bits)} & \multicolumn{10}{l|}{Address (9 bit)} & \multicolumn{2}{l|}{op2 (2 bit)} & \multicolumn{5}{l|}{Rn (5 bit)} & \multicolumn{5}{l|}{Rd (5 bit)} & \multicolumn{3}{l|}{} \\ \cline{1-31}
I (Immediate) & \multicolumn{8}{l|}{opcode (8 bits)} & \multicolumn{12}{l}{Immediate Data (11 bit)} & \multicolumn{5}{l|}{Rn (5 bit)} & \multicolumn{5}{l|}{Rd (5 bit)} & \multicolumn{3}{l|}{} \\ \hline
\end{tabular}
\end{table}
\end{landscape}

\subsection{Instructions}




%=======================================

\subsubsection{Set}
\begin{table}[!h]
\centering
\caption*{SET}
\begin{tabular}{llllll}
ASM & opcode & Format & Description & Operation & ALU Cycles \\ \hline
\multicolumn{1}{|c|}{SET} & \multicolumn{1}{c|}{00001} & \multicolumn{1}{c|}{I} & \DescEntry{Sets Ri to given floating point value} \vline & \multicolumn{1}{c|}{Ri $\leftarrow$  FPvalue} & \multicolumn{1}{c|}{1} \TBstrut \\[1em] \hline
\end{tabular}
\end{table}

\begin{itemize}
    \setlength{\parskip}{0pt}
    \setlength{\itemsep}{0pt plus 1pt}
    \setlength{\itemindent}{-4mm}
    \item[] \textbf{ASM Example:} Set Ri, \#FPvalue
\end{itemize}
\begin{itemize}
    \setlength{\parskip}{0pt}
    \setlength{\itemsep}{0pt plus 1pt}
    \setlength{\itemindent}{7mm}
    \item [\textbf{Flags}]
    \item Zero
    \item Negitive
    \item Error
\end{itemize}

\subsubsection{Load}
\begin{table}[!h]
\centering
\caption*{LOD}
\begin{tabular}{llllll}
ASM & opcode & Format & Description & Operation & ALU Cycles \\ \hline
\multicolumn{1}{|c|}{LOD} & \multicolumn{1}{c|}{00010} & \multicolumn{1}{c|}{D} & \DescEntry{Copies Rj from memory and into Ri} \vline & \multicolumn{1}{c|}{Ri $\leftarrow$  M[Rj]} & \multicolumn{1}{c|}{1} \TBstrut \\[1em] \hline
\end{tabular}
\end{table}

\begin{itemize}
    \setlength{\parskip}{0pt}
    \setlength{\itemsep}{0pt plus 1pt}
    \setlength{\itemindent}{-4mm}
    \item[] \textbf{ASM Example:} Load Ri, Rj
\end{itemize}
\begin{itemize}
    \setlength{\parskip}{0pt}
    \setlength{\itemsep}{0pt plus 1pt}
    \setlength{\itemindent}{7mm}
    \item [\textbf{Flags}]
    \item Zero
    \item Negitive
    \item Error
\end{itemize}

\subsubsection{Store}
\begin{table}[!h]
\centering
\caption*{STR}
\begin{tabular}{llllll}
ASM & opcode & Format & Description & Operation & ALU Cycles \\ \hline
\multicolumn{1}{|c|}{STR} & \multicolumn{1}{c|}{00011} & \multicolumn{1}{c|}{D} & \DescEntry{Copies data from register Rj into memory} \vline & \multicolumn{1}{c|}{M[Ri] $\leftarrow$  Rj} & \multicolumn{1}{c|}{1} \TBstrut \\[1em] \hline
\end{tabular}
\end{table}

\begin{itemize}
    \setlength{\parskip}{0pt}
    \setlength{\itemsep}{0pt plus 1pt}
    \setlength{\itemindent}{-4mm}
    \item[] \textbf{ASM Example:} Store Ri, Rj
\end{itemize}
\begin{itemize}
    \setlength{\parskip}{0pt}
    \setlength{\itemsep}{0pt plus 1pt}
    \setlength{\itemindent}{7mm}
    \item [\textbf{Flags}]
    \item Zero
    \item Negitive
    \item Error
\end{itemize}

\newpage

\subsubsection{Move}
\begin{table}[!h]
\centering
\caption*{MOV}
\begin{tabular}{llllll}
ASM & opcode & Format & Description & Operation & ALU Cycles \\ \hline
\multicolumn{1}{|c|}{MOV} & \multicolumn{1}{c|}{00100} & \multicolumn{1}{c|}{R} & \DescEntry{Moves the value of Rj to Ri, deleting the original} \vline & \multicolumn{1}{c|}{Ri $\leftarrow$  Rj} & \multicolumn{1}{c|}{1} \TBstrut \\[1em] \hline
\end{tabular}
\end{table}

\begin{itemize}
    \setlength{\parskip}{0pt}
    \setlength{\itemsep}{0pt plus 1pt}
    \setlength{\itemindent}{-4mm}
    \item[] \textbf{ASM Example:} Move Ri, Rj
\end{itemize}
\begin{itemize}
    \setlength{\parskip}{0pt}
    \setlength{\itemsep}{0pt plus 1pt}
    \setlength{\itemindent}{7mm}
    \item [\textbf{Flags}]
    \item Zero
    \item Negitive
    \item Error
\end{itemize}

\subsubsection{Add}
\begin{table}[!h]
\centering
\caption*{ADD}
\begin{tabular}{llllll}
ASM & opcode & Format & Description & Operation & ALU Cycles \\ \hline
\multicolumn{1}{|c|}{ADD} & \multicolumn{1}{c|}{00101} & \multicolumn{1}{c|}{R} & \DescEntry{Adds Rj and Rk into Ri} \vline & \multicolumn{1}{c|}{Ri $\leftarrow$  Rj + Rk} & \multicolumn{1}{c|}{3} \TBstrut \\[1em] \hline
\end{tabular}
\end{table}

\begin{itemize}
    \setlength{\parskip}{0pt}
    \setlength{\itemsep}{0pt plus 1pt}
    \setlength{\itemindent}{-4mm}
    \item[] \textbf{ASM Example:} Fadd Ri, Rj, Rk
\end{itemize}
\begin{itemize}
    \setlength{\parskip}{0pt}
    \setlength{\itemsep}{0pt plus 1pt}
    \setlength{\itemindent}{7mm}
    \item [\textbf{Flags}]
    \item Zero
    \item Negitive
    \item Overflow
    \item Carry
    \item Error
\end{itemize}

\subsubsection{Subtract}
\begin{table}[!h]
\centering
\caption*{SUB}
\begin{tabular}{llllll}
ASM & opcode & Format & Description & Operation & ALU Cycles \\ \hline
\multicolumn{1}{|c|}{SUB} & \multicolumn{1}{c|}{00110} & \multicolumn{1}{c|}{R} & \DescEntry{Subtrcts Rk from Rj into Ri} \vline & \multicolumn{1}{c|}{Ri $\leftarrow$  Rj – Rk} & \multicolumn{1}{c|}{3} \TBstrut \\[1em] \hline
\end{tabular}
\end{table}

\begin{itemize}
    \setlength{\parskip}{0pt}
    \setlength{\itemsep}{0pt plus 1pt}
    \setlength{\itemindent}{-4mm}
    \item[] \textbf{ASM Example:} Fsub Ri, Rj, Rk
\end{itemize}
\begin{itemize}
    \setlength{\parskip}{0pt}
    \setlength{\itemsep}{0pt plus 1pt}
    \setlength{\itemindent}{7mm}
    \item [\textbf{Flags}]
    \item Zero
    \item Negitive
    \item Overflow
    \item Carry
    \item Error
\end{itemize}

\newpage

\subsubsection{Negate}
\begin{table}[!h]
\centering
\caption*{NEG}
\begin{tabular}{llllll}
ASM & opcode & Format & Description & Operation & ALU Cycles \\ \hline
\multicolumn{1}{|c|}{NEG} & \multicolumn{1}{c|}{00111} & \multicolumn{1}{c|}{R} & \DescEntry{Sets Ri to the Opposite of Rj} \vline & \multicolumn{1}{c|}{Ri $\leftarrow$  -Rj} & \multicolumn{1}{c|}{1} \TBstrut \\[1em] \hline
\end{tabular}
\end{table}

\begin{itemize}
    \setlength{\parskip}{0pt}
    \setlength{\itemsep}{0pt plus 1pt}
    \setlength{\itemindent}{-4mm}
    \item[] \textbf{ASM Example:} Fneg Ri, Rj
\end{itemize}
\begin{itemize}
    \setlength{\parskip}{0pt}
    \setlength{\itemsep}{0pt plus 1pt}
    \setlength{\itemindent}{7mm}
    \item [\textbf{Flags}]
    \item Negitive
    \item Error
\end{itemize}

\subsubsection{Multiply}
\begin{table}[!h]
\centering
\caption*{MUL}
\begin{tabular}{llllll}
ASM & opcode & Format & Description & Operation & ALU Cycles \\ \hline
\multicolumn{1}{|c|}{MUL} & \multicolumn{1}{c|}{01000} & \multicolumn{1}{c|}{R} & \DescEntry{Multiplies Rj and Rk into Ri} \vline & \multicolumn{1}{c|}{Ri $\leftarrow$  Rj * Rk} & \multicolumn{1}{c|}{5} \TBstrut \\[1em] \hline
\end{tabular}
\end{table}

\begin{itemize}
    \setlength{\parskip}{0pt}
    \setlength{\itemsep}{0pt plus 1pt}
    \setlength{\itemindent}{-4mm}
    \item[] \textbf{ASM Example:} Fmul Ri, Rj, Rk
\end{itemize}
\begin{itemize}
    \setlength{\parskip}{0pt}
    \setlength{\itemsep}{0pt plus 1pt}
    \setlength{\itemindent}{7mm}
    \item [\textbf{Flags}]
    \item Zero
    \item Negitive
    \item Overflow
    \item Carry
    \item Error
\end{itemize}

\subsubsection{Divide}
\begin{table}[!h]
\centering
\caption*{DIV}
\begin{tabular}{llllll}
ASM & opcode & Format & Description & Operation & ALU Cycles \\ \hline
\multicolumn{1}{|c|}{DIV} & \multicolumn{1}{c|}{01001} & \multicolumn{1}{c|}{R} & \DescEntry{Divides Rj by Rk into Ri} \vline & \multicolumn{1}{c|}{Ri $\leftarrow$  Rj / Rk} & \multicolumn{1}{c|}{8} \TBstrut \\[1em] \hline
\end{tabular}
\end{table}

\begin{itemize}
    \setlength{\parskip}{0pt}
    \setlength{\itemsep}{0pt plus 1pt}
    \setlength{\itemindent}{-4mm}
    \item[] \textbf{ASM Example:} Fdiv Ri, Rj, Rk
\end{itemize}
\begin{itemize}
    \setlength{\parskip}{0pt}
    \setlength{\itemsep}{0pt plus 1pt}
    \setlength{\itemindent}{7mm}
    \item [\textbf{Flags}]
    \item Zero
    \item Negitive
    \item Overflow
    \item Carry
    \item Error
\end{itemize}

\newpage

\subsubsection{Floor}
\begin{table}[!h]
\centering
\caption*{FLR}
\begin{tabular}{llllll}
ASM & opcode & Format & Description & Operation & ALU Cycles \\ \hline
\multicolumn{1}{|c|}{FLR} & \multicolumn{1}{c|}{01010} & \multicolumn{1}{c|}{R} & \DescEntry{Sets Ri to the floor of Rj} \vline & \multicolumn{1}{c|}{Ri $\leftarrow$  $\lfloor$ Rj$\rfloor$ } & \multicolumn{1}{c|}{1} \TBstrut \\[1em] \hline
\end{tabular}
\end{table}

\begin{itemize}
    \setlength{\parskip}{0pt}
    \setlength{\itemsep}{0pt plus 1pt}
    \setlength{\itemindent}{-4mm}
    \item[] \textbf{ASM Example:} Floor Ri, Rj
\end{itemize}
\begin{itemize}
    \setlength{\parskip}{0pt}
    \setlength{\itemsep}{0pt plus 1pt}
    \setlength{\itemindent}{7mm}
    \item [\textbf{Flags}]
    \item Zero
    \item Negitive
    \item Error
\end{itemize}

\subsubsection{Ceiling}
\begin{table}[!h]
\centering
\caption*{CEL}
\begin{tabular}{llllll}
ASM & opcode & Format & Description & Operation & ALU Cycles \\ \hline
\multicolumn{1}{|c|}{CEL} & \multicolumn{1}{c|}{01011} & \multicolumn{1}{c|}{R} & \DescEntry{Seting Ri to the ceil of Rj} \vline & \multicolumn{1}{c|}{Ri $\leftarrow$  $\lceil$ Rj$\rceil$ } & \multicolumn{1}{c|}{1} \TBstrut \\[1em] \hline
\end{tabular}
\end{table}

\begin{itemize}
    \setlength{\parskip}{0pt}
    \setlength{\itemsep}{0pt plus 1pt}
    \setlength{\itemindent}{-4mm}
    \item[] \textbf{ASM Example:} Ceil Ri, Rj
\end{itemize}
\begin{itemize}
    \setlength{\parskip}{0pt}
    \setlength{\itemsep}{0pt plus 1pt}
    \setlength{\itemindent}{7mm}
    \item [\textbf{Flags}]
    \item Zero
    \item Negitive
    \item Error
\end{itemize}

\subsubsection{Round}
\begin{table}[!h]
\centering
\caption*{RND}
\begin{tabular}{llllll}
ASM & opcode & Format & Description & Operation & ALU Cycles \\ \hline
\multicolumn{1}{|c|}{RND} & \multicolumn{1}{c|}{01100} & \multicolumn{1}{c|}{R} & \DescEntry{Sets Ri to Rj rounded to the nearest whole number} \vline & \multicolumn{1}{c|}{Ri $\leftarrow$  round(Rj)} & \multicolumn{1}{c|}{1} \TBstrut \\[1em] \hline
\end{tabular}
\end{table}

\begin{itemize}
    \setlength{\parskip}{0pt}
    \setlength{\itemsep}{0pt plus 1pt}
    \setlength{\itemindent}{-4mm}
    \item[] \textbf{ASM Example:} Round Ri, Rj
\end{itemize}
\begin{itemize}
    \setlength{\parskip}{0pt}
    \setlength{\itemsep}{0pt plus 1pt}
    \setlength{\itemindent}{7mm}
    \item [\textbf{Flags}]
    \item Zero
    \item Negitive
    \item Overflow
    \item Carry
    \item Error
\end{itemize}

\newpage

\subsubsection{Absolute Value}
\begin{table}[!h]
\centering
\caption*{ABS}
\begin{tabular}{llllll}
ASM & opcode & Format & Description & Operation & ALU Cycles \\ \hline
\multicolumn{1}{|c|}{ABS} & \multicolumn{1}{c|}{01101} & \multicolumn{1}{c|}{R} & \DescEntry{Sets Ri to the absolute value of Rj} \vline & \multicolumn{1}{c|}{Ri $\leftarrow$  | Rj |} & \multicolumn{1}{c|}{1} \TBstrut \\[1em] \hline
\end{tabular}
\end{table}

\begin{itemize}
    \setlength{\parskip}{0pt}
    \setlength{\itemsep}{0pt plus 1pt}
    \setlength{\itemindent}{-4mm}
    \item[] \textbf{ASM Example:} Fabs Ri, Rj
\end{itemize}
\begin{itemize}
    \setlength{\parskip}{0pt}
    \setlength{\itemsep}{0pt plus 1pt}
    \setlength{\itemindent}{7mm}
    \item [\textbf{Flags}]
    \item Zero
    \item Error
\end{itemize}

\subsubsection{Minimum}
\begin{table}[!h]
\centering
\caption*{MIN}
\begin{tabular}{llllll}
ASM & opcode & Format & Description & Operation & ALU Cycles \\ \hline
\multicolumn{1}{|c|}{MIN} & \multicolumn{1}{c|}{01110} & \multicolumn{1}{c|}{R} & \DescEntry{Sets Ri to the minimum value between Rj and Rk?} \vline & \multicolumn{1}{c|}{Ri $\leftarrow$  min( Rj, Rk)} & \multicolumn{1}{c|}{1} \TBstrut \\[1em] \hline
\end{tabular}
\end{table}

\begin{itemize}
    \setlength{\parskip}{0pt}
    \setlength{\itemsep}{0pt plus 1pt}
    \setlength{\itemindent}{-4mm}
    \item[] \textbf{ASM Example:} Min Ri, Rj, Rk
\end{itemize}
\begin{itemize}
    \setlength{\parskip}{0pt}
    \setlength{\itemsep}{0pt plus 1pt}
    \setlength{\itemindent}{7mm}
    \item [\textbf{Flags}]
    \item Zero
    \item Negitive
    \item Error
\end{itemize}

\subsubsection{Maximum}
\begin{table}[!h]
\centering
\caption*{MAX}
\begin{tabular}{llllll}
ASM & opcode & Format & Description & Operation & ALU Cycles \\ \hline
\multicolumn{1}{|c|}{MAX} & \multicolumn{1}{c|}{01111} & \multicolumn{1}{c|}{R} & \DescEntry{Sets Ri to the maximum value between Rj and Rk?} \vline & \multicolumn{1}{c|}{Ri $\leftarrow$  max( Rj, Rk)} & \multicolumn{1}{c|}{1} \TBstrut \\[1em] \hline
\end{tabular}
\end{table}

\begin{itemize}
    \setlength{\parskip}{0pt}
    \setlength{\itemsep}{0pt plus 1pt}
    \setlength{\itemindent}{-4mm}
    \item[] \textbf{ASM Example:} Max Ri, Rj, Rk
\end{itemize}
\begin{itemize}
    \setlength{\parskip}{0pt}
    \setlength{\itemsep}{0pt plus 1pt}
    \setlength{\itemindent}{7mm}
    \item [\textbf{Flags}]
    \item Zero
    \item Negitive
    \item Error
\end{itemize}

\newpage

\subsubsection{Power}
\begin{table}[!h]
\centering
\caption*{POW}
\begin{tabular}{llllll}
ASM & opcode & Format & Description & Operation & ALU Cycles \\ \hline
\multicolumn{1}{|c|}{POW} & \multicolumn{1}{c|}{10000} & \multicolumn{1}{c|}{I} & \DescEntry{Sets Ri to Rj raised to some given integer power} \vline & \multicolumn{1}{c|}{Ri $\leftarrow$  Rj\textasciicircum integer\_value} & \multicolumn{1}{c|}{6} \TBstrut \\[1em] \hline
\end{tabular}
\end{table}

\begin{itemize}
    \setlength{\parskip}{0pt}
    \setlength{\itemsep}{0pt plus 1pt}
    \setlength{\itemindent}{-4mm}
    \item[] \textbf{ASM Example:} Pow Ri, Rj, \#integer\_value
\end{itemize}
\begin{itemize}
    \setlength{\parskip}{0pt}
    \setlength{\itemsep}{0pt plus 1pt}
    \setlength{\itemindent}{7mm}
    \item [\textbf{Flags}]
    \item Zero
    \item Negitive
    \item Overflow
    \item Carry
    \item Error
\end{itemize}

\subsubsection{Exponent}
\begin{table}[!h]
\centering
\caption*{EXP}
\begin{tabular}{llllll}
ASM & opcode & Format & Description & Operation & ALU Cycles \\ \hline
\multicolumn{1}{|c|}{EXP} & \multicolumn{1}{c|}{10001} & \multicolumn{1}{c|}{R} & \DescEntry{Sets Ri to Rj exponentiated} \vline & \multicolumn{1}{c|}{Ri $\leftarrow$  e\textasciicircum Rj} & \multicolumn{1}{c|}{8} \TBstrut \\[1em] \hline
\end{tabular}
\end{table}

\begin{itemize}
    \setlength{\parskip}{0pt}
    \setlength{\itemsep}{0pt plus 1pt}
    \setlength{\itemindent}{-4mm}
    \item[] \textbf{ASM Example:} Exp Ri, Rj
\end{itemize}
\begin{itemize}
    \setlength{\parskip}{0pt}
    \setlength{\itemsep}{0pt plus 1pt}
    \setlength{\itemindent}{7mm}
    \item [\textbf{Flags}]
    \item Overflow
    \item Carry
    \item Error
\end{itemize}

\subsubsection{Square Root}
\begin{table}[!h]
\centering
\caption*{SQR}
\begin{tabular}{llllll}
ASM & opcode & Format & Description & Operation & ALU Cycles \\ \hline
\multicolumn{1}{|c|}{SQR} & \multicolumn{1}{c|}{10010} & \multicolumn{1}{c|}{R} & \DescEntry{Sets Ri to the square root of Rj} \vline & \multicolumn{1}{c|}{Ri $\leftarrow$  $\surd$ Rj} & \multicolumn{1}{c|}{8} \TBstrut \\[1em] \hline
\end{tabular}
\end{table}

\begin{itemize}
    \setlength{\parskip}{0pt}
    \setlength{\itemsep}{0pt plus 1pt}
    \setlength{\itemindent}{-4mm}
    \item[] \textbf{ASM Example:} Sqrt Ri, Rj
\end{itemize}
\begin{itemize}
    \setlength{\parskip}{0pt}
    \setlength{\itemsep}{0pt plus 1pt}
    \setlength{\itemindent}{7mm}
    \item [\textbf{Flags}]
    \item Zero
    \item Overflow
    \item Carry
    \item Error
\end{itemize}

\newpage

\subsubsection{Branch (Uncond.)}
\begin{table}[!h]
\centering
\caption*{BRU}
\begin{tabular}{llllll}
ASM & opcode & Format & Description & Operation & ALU Cycles \\ \hline
\multicolumn{1}{|c|}{BRU} & \multicolumn{1}{c|}{10011} & \multicolumn{1}{c|}{} & \DescEntry{Loads Ri from memory into PC} \vline & \multicolumn{1}{c|}{PC $\leftarrow$  M[Ri]} & \multicolumn{1}{c|}{1} \TBstrut \\[1em] \hline
\end{tabular}
\end{table}

\begin{itemize}
    \setlength{\parskip}{0pt}
    \setlength{\itemsep}{0pt plus 1pt}
    \setlength{\itemindent}{-4mm}
    \item[] \textbf{ASM Example:} B Ri
\end{itemize}
\begin{itemize}
    \setlength{\parskip}{0pt}
    \setlength{\itemsep}{0pt plus 1pt}
    \setlength{\itemindent}{7mm}
    \item [\textbf{Flags}]
    \item Error
\end{itemize}

\subsubsection{Branch Zero}
\begin{table}[!h]
\centering
\caption*{BRZ}
\begin{tabular}{llllll}
ASM & opcode & Format & Description & Operation & ALU Cycles \\ \hline
\multicolumn{1}{|c|}{BRZ} & \multicolumn{1}{c|}{10100} & \multicolumn{1}{c|}{} & \DescEntry{Sends the PC to a specific labeled line if Ri is zero} \vline & \multicolumn{1}{c|}{If (Ri == 0) PC $\leftarrow$  LABEL (line)} & \multicolumn{1}{c|}{3} \TBstrut \\[1em] \hline
\end{tabular}
\end{table}

\begin{itemize}
    \setlength{\parskip}{0pt}
    \setlength{\itemsep}{0pt plus 1pt}
    \setlength{\itemindent}{-4mm}
    \item[] \textbf{ASM Example:} BZ Ri, LABEL
\end{itemize}
\begin{itemize}
    \setlength{\parskip}{0pt}
    \setlength{\itemsep}{0pt plus 1pt}
    \setlength{\itemindent}{7mm}
    \item [\textbf{Flags}]
    \item Zero
    \item Error
\end{itemize}

\subsubsection{Branch Negative}
\begin{table}[!h]
\centering
\caption*{BRN}
\begin{tabular}{llllll}
ASM & opcode & Format & Description & Operation & ALU Cycles \\ \hline
\multicolumn{1}{|c|}{BRN} & \multicolumn{1}{c|}{10101} & \multicolumn{1}{c|}{} & \DescEntry{Sends the PC to a specific labeled line if Ri is negative} \vline & \multicolumn{1}{c|}{If (Ri < 0) PC $\leftarrow$  LABEL (line)} & \multicolumn{1}{c|}{3} \TBstrut \\[1em] \hline
\end{tabular}
\end{table}

\begin{itemize}
    \setlength{\parskip}{0pt}
    \setlength{\itemsep}{0pt plus 1pt}
    \setlength{\itemindent}{-4mm}
    \item[] \textbf{ASM Example:} BN Ri, LABEL
\end{itemize}
\begin{itemize}
    \setlength{\parskip}{0pt}
    \setlength{\itemsep}{0pt plus 1pt}
    \setlength{\itemindent}{7mm}
    \item [\textbf{Flags}]
    \item Negitive
    \item Error
\end{itemize}

\newpage

\subsubsection{No-op}
\begin{table}[!h]
\centering
\caption*{NOP}
\begin{tabular}{llllll}
ASM & opcode & Format & Description & Operation & ALU Cycles \\ \hline
\multicolumn{1}{|c|}{NOP} & \multicolumn{1}{c|}{10110} & \multicolumn{1}{c|}{} & \DescEntry{No operation} \vline & \multicolumn{1}{c|}{No operation} & \multicolumn{1}{c|}{1} \TBstrut \\[1em] \hline
\end{tabular}
\end{table}

\begin{itemize}
    \setlength{\parskip}{0pt}
    \setlength{\itemsep}{0pt plus 1pt}
    \setlength{\itemindent}{-4mm}
    \item[] \textbf{ASM Example:} Nop
\end{itemize}
\begin{itemize}
    \setlength{\parskip}{0pt}
    \setlength{\itemsep}{0pt plus 1pt}
    \setlength{\itemindent}{7mm}
    \item [\textbf{Flags}]
    \item Error
\end{itemize}

\subsubsection{Halt}
\begin{table}[!h]
\centering
\caption*{HLT}
\begin{tabular}{llllll}
ASM & opcode & Format & Description & Operation & ALU Cycles \\ \hline
\multicolumn{1}{|c|}{HLT} & \multicolumn{1}{c|}{10111} & \multicolumn{1}{c|}{} & \DescEntry{Stop program} \vline & \multicolumn{1}{c|}{Stop Program} & \multicolumn{1}{c|}{-} \TBstrut \\[1em] \hline
\end{tabular}
\end{table}

\begin{itemize}
    \setlength{\parskip}{0pt}
    \setlength{\itemsep}{0pt plus 1pt}
    \setlength{\itemindent}{-4mm}
    \item[] \textbf{ASM Example:} Halt
\end{itemize}
\begin{itemize}
    \setlength{\parskip}{0pt}
    \setlength{\itemsep}{0pt plus 1pt}
    \setlength{\itemindent}{7mm}
    \item [\textbf{Flags}]
    \item Error
\end{itemize}



%========================================

\section{Architecture}
\subsection{Datapath}
\subsection{controller}
\section{VHDL Description}
\section{Testing}
\section{Conclusion}
\end{document}
